\documentclass[12 pt]{beamer}
\usetheme[
	bullet=circle,		% Other option: square
	bigpagenumber,		% circled page number on lower right
	topline=true,			% colored bar at the top of the frame 
	shadow=false,			% Shading for beamer blocks
	watermark=BG_lower,	% png file for the watermark
	]{Flip}


\newcommand{\titleimage}{title}			% Custom title 
\newcommand{\tanedo}{tanedolight}		% Custom author name
\newcommand{\CMSSMDM}{CMSSMDMlight.png}	% light background plot


%%%%%%%%%%
% FONTS %
%%%%%%%%%%

\usepackage[T1]{fontenc}
%\usepackage{lmodern}		
%\usepackage{sfmath}		% Sans Serif Math, off by default

%% Protects fonts from Beamer screwing with them
%% http://tex.stackexchange.com/questions/10488/force-computer-modern-in-math-mode
\usefonttheme{professionalfonts}


\usepackage[no-math]{fontspec}		

%\defaultfontfeatures{Mapping=tex-text}	% This seems to be important for mapping glyphs properly

\usepackage{amsmath}
%\usepackage{amsfonts}
%\usepackage{amssymb}
\usepackage{mathspec}
\usepackage{graphicx}
%\usepackage{mathrsfs} 			% For Weinberg-esque letters
\usepackage{cancel}				% For "SUSY-breaking" symbol
\usepackage{slashed}            % for slashed characters in math mode
%\usepackage{bbm}                % for \mathbbm{1} (unit matrix)
\usepackage{amsthm}				% For theorem environment
\usepackage{multirow}			% For multi row cells in table
\usepackage{arydshln} 			% For dashed lines in arrays and tables
\usepackage{tikzfeynman}		% For Feynman diagrams
% \usepackage{subfig}           % for sub figures
% \usepackage{young}			% For Young Tableaux
% \usepackage{xspace}			% For spacing after commands
% \usepackage{wrapfig}			% for Text wrap around figures
% \usepackage{framed}

\setsansfont{calibri}[ 
Extension = .ttf,
UprightFont = *,
BoldFont = *b,
ItalicFont = *i,
Scale = 1
]

\setmathfont(Digits,Latin,Greek){SitkaI.ttc}

\graphicspath{{images/}}	% Put all images in this directory. Avoids clutter.


\usetikzlibrary{backgrounds}
\usetikzlibrary{mindmap,trees}	% For mind map
% http://www.texample.net/tikz/examples/computer-science-mindmap/


% SOME COMMANDS THAT I FIND HANDY
% \renewcommand{\tilde}{\widetilde} % dinky tildes look silly, dosn't work with fontspec
\newcommand{\comment}[1]{\textcolor{comment}{\footnotesize{#1}\normalsize}} % comment mild
\newcommand{\Comment}[1]{\textcolor{Comment}{\footnotesize{#1}\normalsize}} % comment bold
\newcommand{\COMMENT}[1]{\textcolor{COMMENT}{\footnotesize{#1}\normalsize}} % comment crazy bold
\newcommand{\Alert}[1]{\textcolor{Alert}{#1}} % louder alert
\newcommand{\ALERT}[1]{\textcolor{ALERT}{#1}} % loudest alert
%% "\alert" is already a beamer pre-defined



\author{}
\title{Informatikai Rendszerek Biztonságtechnikája}
\institute{}
\date{}



\begin{document}

%% To use external nodes; http://www.texample.net/tikz/examples/beamer-arrows/
\tikzstyle{every picture}+=[remember picture]

{
\setbeamertemplate{sidebar right}{\llap{\includegraphics[width=\paperwidth,height=\paperheight]{backgnd}}}

\begin{frame}[c]
\begin{center}
	% \includegraphics[width=7cm]{WarpedPenguinsReturn}

  \Large
  \textbf{Információs rendszerek}

  \textbf{biztonságtechnikája}

  \qquad
  
  Bevezetés
  
  \qquad

  \textit{Vakulya Gergely}



	%\begin{tikzpicture}%[show background grid] %% Use grid for positioning, then turn off
	%	\node[inner sep=0pt,above right] (title) 
	%		{ \includegraphics[width=7cm]{\titleimage} };
	%	% \node (title) at (1.5,1.5) {};
	%\end{tikzpicture}
	%\quad

	% \includegraphics[width=7cm]{\titleimage} 
	
	%\vspace{1em}
	%\footnotesize\textcolor{gray}{Journal of Cool Beans
	%\texttt{[arXiv:1234.5678]}}
	%\vspace{.5em}
	
	%\includegraphics[height=1.5cm]{\tanedo} \quad
	 % {\fontspec{Zapfino} Flip Tanedo} \quad
	% \includegraphics[height=1cm]{FlipSansSerif} \quad
	%\includegraphics[height=1.5cm]{CUasym}\\
	% \footnotesize\textcolor{gray}{In collaboration with} Csaba Cs\'aki\textcolor{gray}{,} Yuval Grossman\textcolor{gray}{, and} Yuhsin Tsai\normalsize\\
	%	\footnotesize\textcolor{gray}{In collaboration with 
%		D. Grayson, J. Todd, T. Drake, S. Brown, D. Wayne}\normalsize\\
%	\textcolor{normal text.fg!50!Comment}{\textit{Gotham University}, \today}
	% \textcolor{Comment}{ \;($\pi$ day)}\\
	% \Comment{4 February 2011}
\end{center}
\end{frame}
}

%------------------------------------------------

%\setbeamertemplate{default}{}

\begin{frame}{Követelmények}

  \begin{block}{2 db ZH}
    \begin{itemize}
        \item{6. hét (március 19.)}
        \item{13. hét (május 7.)}
        \item{Kb. 60 perces feladatsor.}
        \item{Pótlási lehetőség: 14. hét (május 14.)}
    \end{itemize}
  \end{block}

  \begin{block}{Beszámoló}
    \begin{itemize}
      \item{Rövid prezentáció önállóan feldolgozott informatikai biztonsági témából}
    \end{itemize}
  \end{block}

  \begin{block}{Elmaradó órák}
    \begin{itemize}
      \item{Március 26. (kari tavaszi túra)}
      \item{Április 2. (rektori szünet)}
      \item{Április 9. (atomerőmű látogatás)}
    \end{itemize}
  \end{block}

\end{frame}

%------------------------------------------------

\begin{frame}{Érintett témák}
  \begin{block}{Kriptográfia}
    \begin{itemize}
      \item{Szimmetrikus kulcsú algoritmusok}
      \item{Nyilvános kulcsú algoritmusok}
      \item{Üzenet pecsét algoritmusok}
      \item{Digitális aláírás}
      \item{Titkosított adattárolás}
      \item{Titkosított kommunikáció}
    \end{itemize}
  \end{block}

  \begin{block}{Sebezhetőségek}
    \begin{itemize}
      \item{Sebezhetőségek típusai}
      \item{Konkrét példák}
      \item{CVE adatbázis}
      \item{Védekezési lehetőségek}
    \end{itemize}
  \end{block}
\end{frame}

%------------------------------------------------

\begin{frame}{A kriptográfia előzményei}
  \begin{itemize}
    \item{Motiváció}
      \begin{itemize}
        \item{Üzenetek átvitele nem biztonságos csatornán}
          \begin{itemize}
            \item{Kommunikáció háborús körülmények között}
            \item{Magánjellegű kommunikáció (szerelmes levelek...)}
          \end{itemize}
        \item{Üzenetek elrejtése mindenki más elől}
          \begin{itemize}
            \item{Titkos napló}
          \end{itemize}
      \end{itemize}

    \item{Megoldási lehetőségek}
      \begin{itemize}
        \item{Lakattal lezárt tároló}
        \item{Lepecsételt boríték}
        \item{\textbf{Szteganográfia}}
      \end{itemize}
  \end{itemize}
\end{frame}

%------------------------------------------------

\begin{frame}{Szteganográfia a kezdetekben}
  \begin{itemize}
    \item{Jelentése: leplezni (ógörög eredetű)}
    \item{Az információt nem rejtjelezzük, hanem elrejtik egy szokványos adathalmazban, amire senki sem figyel.}
    \item{Az adat elrejtése az adatban}
    \item{Az ókortól kezdve alkalmazzák}
  \end{itemize}

  \begin{block}{Kínai módszer}
    A hírvivő lenyelte a gombócba gyúrt, viaszba mártott levelet.
  \end{block}

  \begin{block}{Rabszolga fejére írt szöveg}
    A rabszolga fejét kopaszta borotválták és arra írták a szöveget. Amikor a haja kinőtt, elküldték a címzetthez, ahol újra leborotválták a haját.
  \end{block}

\end{frame}

%------------------------------------------------

\begin{frame}{Szteganográfia a kezdetekben}

  \begin{block}{Viasztáblás módszer}
    A viasztábláról lekaparták a viaszt és a csupasz deszkára vésték a szöveget. Ezután visszahelyezték a viaszt.
  \end{block}

  \begin{block}{Dupla levél}
    A hírvivő egy állevelet visz, de van nála (elrejtve, pl. lábbelibe varrva) egy másik levél is.
  \end{block}

  \begin{block}{Halhólyag}
    A felfújt hólyagra írják az üzenetet, amit aztán leeresztenek. Elolvasáskor újra fel kell fújni.
  \end{block}

  \begin{block}{Láthatatlan tinta}
    A klasszikus módszer. Régóta ismertek hozzá megfelelő anyagok.
  \end{block}
  
\end{frame}

%------------------------------------------------

\begin{frame}{Versbe szedett üzenet}
  \begin{block}{Gárdonyi Géza: Egy magyar rab levele}

\centering
,,Kedves ezüstös, drága dádém!
Ezer nemes arany tizedét örömmel ropogtasd
örök keserűség keservét ivó magzatodért.
Egészségem gyöngy. A vaj árt. Ritkán óhajtom
sóval, borssal. Ócska lepedőben szárítkozom
álmomban, zivataros estén. Matyi
bátyám, egypár rózsát, rezet, ezüstöt, libát
egy lapos leveleddel eressze hajlékomba.
Erzsi, tűt, faggyút, ollót, gombot, levendulát adj!
Laci, nefelejts!

\qquad

\flushright Imre''
  \end{block}


Próbáljuk meg megfejteni!

\end{frame}

%------------------------------------------------

\begin{frame}{Modern kori szteganográfia}
  Általában egy fájlban rejtenek el egy másik fájlt.

  \begin{block}{Camouflage}
    A Camouflage egy olyan program, amivel bármilyen fájlt például képfájlba, vagy Word dokumentumba lehet rejteni. Zeneszámok terjesztésére használták.
  \end{block}

  \begin{block}{Kép a képben}
    Különböző megoldások, amikkel egy kép pixeleinek minimális módosításával rejthetőek el adatok. Inkább érdekesség.
  \end{block}

\end{frame}


%------------------------------------------------

\begin{frame}{A kriptográfia múltja}
  \begin{block}{A szó eredete}
    \begin{itemize}
      \item{Szintén ógörög}
      \item{\textit{kryptós}: rejtett}
      \item{\textit{gráphein}: írni}
      \item{\textbf{titkosírás}}
    \end{itemize}
  \end{block}

  \begin{block}{Alapfogalmak}
    \begin{itemize}
      \item{Kriptográfia: információrejtés, rejtjelezés}
      \item{Kriptoanalízis: visszafejtés}
      \item{Kriptológia: Mind a kriptográfiát, mind a kriptoanalízist magában foglaló tudomány}
    \end{itemize}
  \end{block}

\end{frame}

%------------------------------------------------

\begin{frame}{Ókor}
  \begin{block}{Spártai bőrszíj}
    \begin{itemize}
      \item{Egy bőrszíjat egy megadott átmérőjű hengerre tekerték fel.}
      \item{A henger alkotói mentén írták fel a szöveget.}
      \item{A letekert szíjon a betűk sorrendje összekeveredett.}
      \item{A dekódoláshoz egy azonos átmérűjű henger kellett.}
    \end{itemize}
  \end{block}

  \begin{block}{Caesar-kód}
    \begin{itemize}
      \item{Minden betű helyett az ABC-ben 3-mal odébb levő betűt írták.}
      \item{Általánosított Caesar-kód: 3 helyett $k$ hellyel odébb csúsztatták az ABC-t.}
    \end{itemize}
  \end{block}

\end{frame}

%------------------------------------------------

\begin{frame}{Egyéb módszerek}
  \begin{block}{Rejtjel-rács}
    \begin{itemize}
      \item{Egy négyzet alakú rácsot használnak, amin kivágások vannak.}
      \item{A kilátszó betűket kell felülről lefelé olvasni.}
    \end{itemize}
  \end{block}

  \begin{block}{Egybetű-helyettesítés}
    \begin{itemize}
      \item{Minden betűt egy másiknak feleltetnek meg az ABC-ben.}
      \item{A Caesar-kód egy speciális Egybetű-helyettesítés.}
    \end{itemize}
  \end{block}

  \begin{block}{Kódkönyv}
    \begin{itemize}
      \item{Az elküldött számok egy könyv oldalainak, szavainak felelnek meg.}
    \end{itemize}
  \end{block}

\end{frame}

%------------------------------------------------

\begin{frame}{Kriptoanalízis}
  \begin{block}{Egybetű-helyettesítéses kódolók}
    \begin{itemize}
      \item{Egy $n$ betűs ABC esetében $n!$ féle egybetű-helyettesítés készíthető.}
      \item{A nyers erő módszerével a próbálkozás elvileg reménytelen.}
      \item{Megoldás: A természetes nyelvek statisztikai tulajdonságai. Gyakoriság-elemzés. Betűpárok, betűhármasok keresése.}
    \end{itemize}
  \end{block}

  \begin{exampleblock}{Próbáljuk meg megfejteni}
   í iak üökoíhaúú pu ub í iak pmvbznwom í euibg ibíodwíh í iak őíkdk íz akmo hézemkmp b íz mpámr iakcí m hézeab píví 
    %a cél voltaképp mi is a cél megszűnte a dicső csatának a cél halál az élet küzdelem s az ember célja e küzdés maga
    %sed 'y/aábcdeéfghiíjklmnoóöőpqrstuúüűvwxy/ídáiemayvőuschkpwöfűgútrboxqénüójl/'
  \end{exampleblock}
\end{frame}

%------------------------------------------------

\end{document}


%db={}
%szoveg="í iak üökoíhaúú pu ub í iak pmvbznwom í euibg ibíodwíh í iak őíkdk íz akmo hézemkmp b íz mpámr iakcí m hézeab píví"
%for element in szoveg:
%  try:
%    db[element] += 1
%  except KeyError:
%    db[element] = 1
%
%for element in db:
%  print(element, db[element])
%
%szoveg = szoveg.replace("í", "A")
%print(szoveg)
