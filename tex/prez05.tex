\documentclass[12 pt]{beamer}
\usetheme[
  bullet=circle,		% Other option: square
  bigpagenumber,		% circled page number on lower right
  topline=true,			% colored bar at the top of the frame 
  shadow=false,			% Shading for beamer blocks
  watermark=BG_lower,	% png file for the watermark
]{Flip}


\newcommand{\titleimage}{title}			% Custom title 
\newcommand{\tanedo}{tanedolight}		% Custom author name
\newcommand{\CMSSMDM}{CMSSMDMlight.png}	% light background plot


%%%%%%%%%%
% FONTS %
%%%%%%%%%%

\usepackage[T1]{fontenc}
%\usepackage{lmodern}		
%\usepackage{sfmath}		% Sans Serif Math, off by default

%% Protects fonts from Beamer screwing with them
%% http://tex.stackexchange.com/questions/10488/force-computer-modern-in-math-mode
\usefonttheme{professionalfonts}


\usepackage[no-math]{fontspec}		

%\defaultfontfeatures{Mapping=tex-text}	% This seems to be important for mapping glyphs properly

\usepackage{amsmath}
%\usepackage{amsfonts}
%\usepackage{amssymb}
\usepackage{mathspec}
\usepackage{graphicx}
%\usepackage{mathrsfs} 			% For Weinberg-esque letters
\usepackage{cancel}				% For "SUSY-breaking" symbol
\usepackage{slashed}            % for slashed characters in math mode
%\usepackage{bbm}                % for \mathbbm{1} (unit matrix)
\usepackage{amsthm}				% For theorem environment
\usepackage{multirow}			% For multi row cells in table
\usepackage{arydshln} 			% For dashed lines in arrays and tables
\usepackage{tikzfeynman}		% For Feynman diagrams
% \usepackage{subfig}           % for sub figures
% \usepackage{young}			% For Young Tableaux
% \usepackage{xspace}			% For spacing after commands
% \usepackage{wrapfig}			% for Text wrap around figures
% \usepackage{framed}

\usepackage{setspace}

\setsansfont{calibri}[ 
Extension = .ttf,
UprightFont = *,
BoldFont = *b,
ItalicFont = *i,
Scale = 1
]

\setmathfont(Digits,Latin,Greek){SitkaI.ttc}

\graphicspath{{images/}}	% Put all images in this directory. Avoids clutter.


\usetikzlibrary{backgrounds}
\usetikzlibrary{mindmap,trees}	% For mind map
\usetikzlibrary{arrows,positioning,calc}
\usetikzlibrary{shapes}
% http://www.texample.net/tikz/examples/computer-science-mindmap/


% SOME COMMANDS THAT I FIND HANDY
% \renewcommand{\tilde}{\widetilde} % dinky tildes look silly, dosn't work with fontspec
\newcommand{\comment}[1]{\textcolor{comment}{\footnotesize{#1}\normalsize}} % comment mild
\newcommand{\Comment}[1]{\textcolor{Comment}{\footnotesize{#1}\normalsize}} % comment bold
\newcommand{\COMMENT}[1]{\textcolor{COMMENT}{\footnotesize{#1}\normalsize}} % comment crazy bold
\newcommand{\Alert}[1]{\textcolor{Alert}{#1}} % louder alert
\newcommand{\ALERT}[1]{\textcolor{ALERT}{#1}} % loudest alert
%% "\alert" is already a beamer pre-defined



\author{}
\title{Informatikai Rendszerek Biztonságtechnikája}
\institute{}
\date{}



\begin{document}

%% To use external nodes; http://www.texample.net/tikz/examples/beamer-arrows/
\tikzstyle{every picture}+=[remember picture]

{
  \setbeamertemplate{sidebar right}{\llap{\includegraphics[width=\paperwidth,height=\paperheight]{backgnd}}}

  \begin{frame}[c]
    \begin{center}
      % \includegraphics[width=7cm]{WarpedPenguinsReturn}

      \Large
      \textbf{Információs rendszerek}

      \textbf{biztonságtechnikája}

      \qquad

      Sebezhetőségek

      \qquad

      \textit{Vakulya Gergely}

    \end{center}
  \end{frame}
}

  %------------------------------------------------

\begin{frame}{A sebezhetőségek}
  \begin{itemize}
    \item{A sebezhetőség bizonyos rendszerek, szoftverek gyengesége, amit egy támadó a saját céljaira, illetve károkozásra tud kihasználni.}
    \item{A sebezhetőséget kihasználó szoftver az exploit.}
    \item{A következők időbeli alakulása lényegében tetszőlegesen alakulhat:}
      \begin{itemize}
        \item{A sebezhetőségek fennállásának időintervalluma}
        \item{Azok felfedezése}
        \item{A javítások elkészítése (a sebezhetőség befoltozása)}
        \item{A sebezhető rendszerek tényleges javítása}
        \item{A sebezhetőség nyilvánosságra hozása}
        \item{A sebezhetőség ismert, vagy feltételezett kihasználása}
      \end{itemize}
  \end{itemize}
\end{frame}

\begin{frame}{A CVE adatbázis}
  \begin{itemize}
    \item{CVE: Common Vulnerabilities and Exposures}
    \item{Nyilvános sérülékenység-adatbázis.}
    \item{A publikált sérülékenységek egy CVE számot kapnak, amiről egyértelműen be lehet őket azonosítani (emellett lehet fantázianevük is).}
    \item{Legfontosabb információk:}
      \begin{itemize}
        \item{Milyen szoftvert (vagy egyéb rendszert érint)}
        \item{Milyen verziókat érint}
        \item{Rögzítés dátuma}
        \item{Leírás (működési mechanizmus)}
        \item{Referenciák (például az egyes szállítók információihoz)}
      \end{itemize}
        \item{Több gyűjtőoldal:}
          \begin{itemize}
            \item{\href{https://cve.mitre.org}{https://cve.mitre.org}}
            \item{\href{https://www.cvedetails.com}{https://www.cvedetails.com}}
            \item{\href{https://www.cve.org/}{https://www.cve.org/}}
            \item{\href{https://nvd.nist.gov}{https://nvd.nist.gov}}
          \end{itemize}
  \end{itemize}
\end{frame}

\begin{frame}{A sebezhetőségek fő típusai}
  \begin{itemize}
    \item{Kódfuttatás (\textbf{code execution}): A támadó (esetenként tetszőleges) kódot képes futtatni a sebezhető rendszeren.}
    \item{Védelem, vagy ellenőrzés megkerülése (\textbf{bypass}): A támadó valamilyen ellenőrzés kihagyására képes a sebezhető rendszert késztetni.}
    \item{Privilégiumszint emelés (\textbf{privilege escalation}): A támadó egy számára nem biztosított privilégiumhoz (például root joghoz) jut.}
    \item{Szolgáltatásmegtagadás (\textbf{denial of service, DoS}): A támadó a rendszer szolgáltatásait leállítja, lelassítja, vagy azokat valamilyen módon akadályozza.}
    \item{Adatszivárgás (\textbf{information leak}): A támadó számára nem nyilvános adatokhoz jut.}
  \end{itemize}
\end{frame}

\begin{frame}{Támadási mechanizmusok}
  \begin{itemize}
    \item{Puffertúlcsordulásos támadás (buffer overflow, stack buffer overflow)}
    \item{SQL-injektálás (SQL injection, SQLi)}
    \item{Cross-side scripting, XSS}
    \item{Kéréshamisítás (Cross-site request forgery, CSRF vagy XSRF)}
    \item{Versenyhelyzet (Race condition)}
    \item{Sidechannel attack}
    \item{Memory corruption}
    \item{Bemenet ellenőrzés (Input validation)}
    \item{Kriptográfiai támadások}
  \end{itemize}

  \qquad

  Ezek kombinációi is gyakoriak.
\end{frame}

\begin{frame}{CVE-2019-18634, sudo pwfeedback exploit (Linux és egyéb rendszerek)}
  \begin{itemize}
    \item{A sudo 1.7.1 (2009-04-11) és 1.8.26 (2018-11-13) verziói között.}
    \item{A \texttt{sudo} parancs a jelszó bekérésekor nem ad kimenetet.}
    \item{Azonban beállíthatjuk, hogy a jelszó begépelése közben csillagok jelenjenek meg.}
    \item{Ha ez a beállítás fennáll, megfelelően összeállított bemenet beírásával a támadó root jogosultsághoz juthat.}
    \item{Típus: Helyi privilégiumszint növelés (local root).}
    \item{Mechanizmus: Stack buffer overflow.}
    \item{Megjegyzés: A `sudo` root joggal fut.}
  \end{itemize}
\end{frame}

%\begin{frame}{CVE-2021-3156, sudo}
%\end{frame}

\begin{frame}{CVE-2016-5195, Dirty COW (Linux)}
  \begin{itemize}
    \item{Linux kernel, 2.x-4.x, a 4.8.3 előtt.}
    \item{COW jelentése: Copy On Write}
    \item{Versenyhelyzet (race condition) alapú sebezhetőség.}
    \item{A kihasználásával a támadó egy olyan, a root tulajdonában levő fájlt tud írni, amire csak olvasási joga van.}
      \begin{itemize}
        \item{Átírhat konfigurációs fájlokat, például jogosultságokat adva magának.}
        \item{Backdoort helyezhet el valamelyik setuid root-os programban.}
      \end{itemize}
  \end{itemize}
\end{frame}

\begin{frame}{A Dirty COW működése}
  \begin{itemize}
    \item{Unix alatt ,,minden file''. A \texttt{/proc} alatt például virtuális fájlok vannak. A \texttt{/proc/self/mem} az aktuális folyamat memóriájának leképezése.}
    \item{\texttt{mmap()} függvény: Egy fájlt a memóriára képez le.}
    \item{Saját COW másolatot is készíthetünk egy fájlról.}
    \item{Copy on write: Csak akkor másolja le ténylegesen, ha módosítjuk.}
    \item{Race condition: Két eseménynak egymáshoz képest speciálisan időzítve kell lefutnia a hibához.}
    \item{Két thread-et futtatunk:}
      \begin{itemize}
        \item{Az egyikben az \texttt{madvice()} függvénnyel azt mondjuk a kernelnek, hogy az adott területet nem fogjuk a közeljövőben használni.}
        \item{A másikban a saját memóriát leképező \texttt{/proc/self/mem} azon részére írunk, ahova a támadott fájlt map-eltük.}
      \end{itemize}
    \item{Nagyon sok próbálkozás után egy bizonyos ponton a kernel ,,elrontja'' az írás és a másolás sorrendjét és felülíródik a megnyitt fájl.}
  \end{itemize}
\end{frame}

\begin{frame}{CVE-2017-5754 (Meltdown) és CVE-2017-5753 / CVE-2017-5715 (Spectre)}
  \begin{itemize}
    \item{Operációs rendszertől független hardveres sebezhetőségek.}
    \item{Sebezhető a Pentium-II-től a 8. generációs Core processzorokig lényegében minden Intel CPU, ezek AMD megfelelői, a modern ARM architektúrák és potenciálisan egyéb CPU-k.}
    \item{Mecanizmus: Race condition, sidechannel attack, timing attack.}
    \item{Következmény: Elvileg nem hozzáférhető memóriaterületek olvasása (például kriptográfiai kulcsok, jelszavak, személyes adatok stb.).}
  \end{itemize}
\end{frame}

\begin{frame}{A modern CPU-k sebességének egyik kulcsa}
  \begin{block}{Pipelining}
    \begin{itemize}
      \item{Az utasítások végrehajtását az órajel vezérli.}
      \item{Naiv megközelítés: 1 órajel alatt 1 utasítás.}
      \item{Az utasítások azonban több lépésből állnak (például: betöltés, dekódolás, végrehajtás).}
      \item{Pipelining: Amíg az egyik utasítást végrehajtja a CPU, a következőt már dekódolja, a rákövetkezőt már betölti.}
      \item{Feltétel: Tudni kell, hogy melyik utasítás következik. Elágazásoknál ez nem egyértelmű.}
    \end{itemize}
  \end{block}

  \begin{block}{Az elágazás problémájának megoldása}
    \begin{itemize}
      \item{Megvárjuk, amíg kiderül, merre folytatódik a program futása}
      \item{Vagy megpróbáljuk megjósolni (branch prediction)}
    \end{itemize}
  \end{block}
\end{frame}

\begin{frame}{Branch prediction és speculatíve execution}
  \begin{block}{Branch prediction}
    \begin{itemize}
      \item{A CPU megjósolja, hogy egy elágazás merre fog folytatódni.}
      \item{Stratégiák:}
        \begin{itemize}
          \item{Mindig az ugrást választja}
          \item{Mindig a nem ugrást választja}
          \item{Azt az ágat választja, ami az előző alkalommal, vagy előző alkalmakkor is végrehajtódott}
        \end{itemize}
    \end{itemize}
  \end{block}
  \begin{block}{Speculative execution}
    \begin{itemize}
      \item{A CPU bizonyos, \textbf{valószínűleg szükséges} utasításokat \textbf{előre} végrehajt.}
      \item{Ha ténylegesen végre kellett hajtania az utasításokat, akkor időt spórolt.}
      \item{Ha mégsem kellett volna végrehajtani, akkor vissza kell csinálni a tévesen végrehajtott műveletet.}
      %\item{A speculatíve execution nyilván branch prediction-nel együtt használatos.}
    \end{itemize}
  \end{block}
\end{frame}

\begin{frame}{A cache memória}
  \begin{itemize}
    \item{Adott időben egy folyamat a memóriának csak egy kis részét használja (lokalitás).}
    \item{A memória elérését a cache gyorsítja.}
    \item{Egy adott memóriarekesz első elérésekor a kérés a memóriából kerül kiszolgálásra és az adat beíródik a cache-be is.}
    \item{Az újabb hivatkozáskor már a sokkal gyorsabb cache-ből lehet kiolvasni az adatot.}
    \item{A cache a memóriához képest nagyságrendekkel kisebb és gyorsabb.}
    \item{A cache általában az LRU (Last Recently Used) algoritmust használja.}
    \item{A cache általában lap alapú, illetve a memórialapozást is gyorsíthatja. Így egy byte beolvasása annak egy lapnyi körnezetét is gyorsítja.}
  \end{itemize}
\end{frame}

\begin{frame}{A sebezhetőség két építőeleme}
  \begin{block}{A branch prediction szándékos félrevezetése}
    \begin{itemize}
      \item{A branch prediction működése szándékosan befolyásolható, vagy megjósolható.}
      \item{Könnyen készíthető olyan elágazás, amit nagy valószínűséggel rosszul fog jósolni a CPU.}
      \item{A branch prediction sok esetben tanítható is.}
    \end{itemize}
  \end{block}

  \begin{block}{Kétszeresen indirekt címzés}
    \texttt{y = a[b[x]]}
    \begin{itemize}
      \item{\texttt{b} egy tömb, aminek az \texttt{y}-adik elemét kiolvassuk.}
%      \item{Az \texttt{a} tömböt ezzel az értékkel indexeljük.}
%      \item{Az \texttt{a} tömbhöz van hozzáférési jogunk, a \texttt{b} tömbhöz nincs.}
    \end{itemize}
  \end{block}
\end{frame}

\begin{frame}{A támadás menete}
  \texttt{if(x < 10)}

  \texttt{\qquad y = a[b[x]]}
 
  \bigskip

  \begin{itemize}
    \item{Az \texttt{a} tömbhöz van hozzáférési jogunk.}
    \item{A \texttt{b} tömb elejéhez van hozzáférési jogunk, a végéhez nincs.}
    \item{Kis értékű \texttt{x}-re a kiolvasás sikeres. Ezekkel tanítjuk a branch prediction-t.}
    \item{Nagy \texttt{x} esetén az eredmény: segmentation fault.}
  \end{itemize}

  \bigskip

  Azonban az elágazás igaz ága spekulatíven lefutott, kiolvasva a nem hozzáférhető területet. Persze mivel
  a feltétel nem teljesült, a processzor visszaállította az eredeti állapotot.

  \begin{alertblock}{}
    Kivéve a cache-t! Most az \texttt{a} tömb egyik eleme benne van a cache-ben, tehát gyorsabban
    kiolvasható.
  \end{alertblock}

\end{frame}

\begin{frame}{Néhány további részlet}
  \begin{block}{A módszer tulajdonságai}
    \begin{itemize}
      \item{Folyamatok egymáshoz képesti lefutásának hibáját használjuk ki, így ez egy \textbf{race condition attack}.}
      \item{Egy olyan csatornán keresztül lehet információhoz jutni, ami nem kötődik konkrétan
        a támadott rendszerhez, szoftverhez, protokollhoz, tehát ez egy \textbf{side-channel attack}.}
      \item{Magát az adatot időzítés segítségével olvassuk ki, így ez egy \textbf{timing attack}.}
    \end{itemize}
  \end{block}

  \begin{block}{Egyéb kiegészítések}
    \begin{itemize}
      \item{Ez így leírt módszer csak a támadás elvét mutatja be, valójában sok konkrét sebezhetőség és támadás létezik.}
      \item{A bemutatott kódrészlet például figyelmen kívül hagyja a lapméretet.}
%      \item{Mivel a cache-ben levő és a nem cache-ben }
    \end{itemize}
  \end{block}
\end{frame}

\begin{frame}{Védekezés a Meltdown / Spectre típusú sebezhetőségekkel szemben}
  \begin{itemize}
    \item{Elvileg teljeskörű javítás csak az érintett CPU cseréjével lehetséges (9. gen. Core CPU-k már védettek).}
    \item{Szoftveres javítás: \textbf{mitigation} (enyhítés, csillapítás), valójában nem }
  \end{itemize}
\end{frame}

\begin{frame}{DoS / DDoS típusú támadások}

\end{frame}

\begin{frame}{Slow loris (CVE-2007-6750 és sok egyéb)}
  \begin{block}{Összesítés}
  \begin{itemize}
    \item{DoS (Denial of Service) típusú támadás webszerverek ellen.}
    \item{Számos webszerver verzió érintett.}
    \item{Inkább konfigurációs jellegű probléma.}
  \end{itemize}
  \end{block}

  \begin{block}{Működése}
  \end{block}

  

\end{frame}


\begin{frame}{CVE-2017-0199, Office arbitrary code execution}
\end{frame}

\begin{frame}{CVE-2017-0144, Eternal Blue}
\end{frame}

\begin{frame}{CVE-2020-0796 aka CoronaBlue aka SMBGhost}
\end{frame}

\begin{frame}{CVE-2023-3269, StackRot}
\end{frame}

\begin{frame}{WiFi routerek jelszavainak megjósolhatóságára vonatkozó sebezhetőségek}
  \begin{itemize}
    \item{A hardver eszközök általában nagy sorozatban készülnek. Egy típuson belül azonosnak tekinthetőek.}
    \item{De mi van, ha mégis különbözniük kell?}
      \begin{itemize}
        \item{MAC cím}
        \item{Egyedi jelszó}
        \item{Egyéb egyedi azonosító}
      \end{itemize}
    \item{A MAC cím egyediségére vannak jól bevált módszerek (maszk ROM)}
    \item{A router háttértára flash memória. Egy firmware frissítés az itt levő jelszót törölné.}
    \item{Ötlet: Származtassuk a jelszót valamilyen egyébként is eltárolt adatból. A WiFi SSID-jét hasonlóképpen.}
  \end{itemize}
\end{frame}

\begin{frame}{WiFi routerek jelszavainak megjósolhatóságára vonatkozó sebezhetőségek}
  \begin{itemize}
    \item{Sebezhetőség:}
      \begin{itemize}
        \item{Ha ismerjük a generátor adatot, kiszámíthatjuk a generált adatokat.}
        \item{A generált adatok között is összefüggés lehet.}
      \end{itemize}
    \item{Két példa:}
  \begin{itemize}
    \item{CVE-2023-47352: Technicolor TC8715D routerek esetén a támadó az SSID és a BSSID (router MAC címe) alapján megjósolhatja a WPA2 jelszót.}
    \item{CVE: N/A, UBEE EVW3226 routerek esetében lényegében ugyanez a sebezhetőségtípus}
    \item{Mindkét router alapértelmezett jelszava admin/admin; az adminisztratív felület elérhető WiFi-ről is}
  \end{itemize}
\item{Megoldási lehetőségek:}
  \begin{itemize}
    \item{Magasabb elvárások a gyártóval kapcsolatban}
    \item{Ne bízzunk az alapértelmezett jelszavakban!}
  \end{itemize}
  \end{itemize}
\end{frame}

\begin{frame}{Authentication bypass típusú sebezhetőségek}
  \begin{itemize}
    \item{Általánosan: A webfelületek nem ellenőrzik minden esetben a felhasználó privilégiumait, azonosságát}
    \item{Példa: Az adminisztratív felületre egy login oldal irányít át; ott már további ellenőrzés nem történik}
    \item{Huawei HG866 Authentication Bypass, CVE: N/A}
    \item{}
  \end{itemize}
\end{frame}

\begin{frame}{CVE-2023-2598, Linux IO uring memory out of bounds read/write}
  \begin{block}{Összefoglaló}
    \begin{itemize}
      \item{CVE-2023-2598}
      \item{Sebezhető szoftver: Linux kernel 6.3-rc1 - 6.4-rc1 (utóbbiban javítva)}
      \item{Mechanizmus: Memory corruption, out of bounds r/w}
      \item{Következmény: Memória szivárgás, privilégiumszint növelés, tetszőleges kódvégrehajtás}
      \item{\href{https://anatomic.rip/cve-2023-2598/}{https://anatomic.rip/cve-2023-2598/}}
    \end{itemize}
  \end{block}
\end{frame}

\begin{frame}{IO uring}
  \begin{block}{Rendszerhívások}
    \begin{itemize}
      \item{Az operációs rendszer szolgáltatásait adó API}
      \item{Jelentős részük valamilyen IO művelet}
      \item{Például fájlműveletek, hálózati kommunikáció}
      \item{A rendszer hívás overheaddel jár}
    \end{itemize}
  \end{block}
  
  \begin{block}{IO uring}
    \begin{itemize}
      \item{Az IO uring megoldás a rendszerhívások által kialakuló overheadet csökkenti}
      \item{Aszinkron működés: nem blokkol}
      \item{Egy körpufferben kell átadni a szükséges rendszerhívásokat, azok paramétereivel}
      \item{A probléma a memóriapufferek átadásánál és használatánál lép fel}
    \end{itemize}
  \end{block}
\end{frame}

\begin{frame}{Néhány szó a Linux memóriaszervezéséről}
  \begin{itemize}
    \item{Lapszervezés: 4 kByte méretű lapokkal}
    \item{A lapszervezés megakadályozza a külső tördelődést, viszont a folytonos területek mégis speciálisan vannak kezelve}
      \begin{itemize}
        \item{Fej (head) lap}
        \item{Farok (tail) lapok}
      \end{itemize}
    \item{Folio-k (ívek)}
      \begin{itemize}
        \item{Az összetett lapok kezelésére használják}
        \item{Egyszerű lap esetén maga a lap}
        \item{Összetett lap esetén a fejlap}
        \item{Semmiképp nem lehet faroklap}
      \end{itemize}
  \end{itemize}
\end{frame}

\begin{frame}{A hiba}
  \begin{block}{A virtuális memória}
    \begin{itemize}
      \item{A virtuális memória lényege, hogy a logikai és fizikai memória között leképezést ad.}
      \item{Egy logikai memórialaphoz egy fizikai lapot lehet rendelni}
    \end{itemize}
  \end{block}

  \begin{block}{A hiba forrása}
    \begin{itemize}
      \item{Ha az IO uring-nek olyan puffer van átadva, ahol egy folytonos logikai terület mindegyik lapjához ugyanaz a fizikai lap van rendelve, akkor a kód elrontja a számolást és az 1 db fizikai lap után következőket is használni fogja}
    \end{itemize}
  \end{block}
\end{frame}

%https://github.com/Snoopy-Sec/Localroot-ALL-CVE/tree/master/2023/CVE-2023-3269
%StackRot (CVE-2023-3269): Linux kernel privilege escalation vulnerability

%Heartbleed CVE-2014-0160
%https://heartbleed.com/

%Log4j
%https://en.wikipedia.org/wiki/Log4j
%https://nvd.nist.gov/vuln/detail/CVE-2021-44228 CVE-2021-44228

%Router privilege escalation
%https://the-infosec.com/2017/03/20/huawei-hg8245h-router-privilege-escalatio/

%Kereso: https://www.shodan.io/search?query=port%3A80+country%3Ahu+org%3A%22DIGI+Tavkozlesi+es+Szolgaltato+Kft.%22

%https://hup.hu/cikkek/20180608/a_vpnfilter_nagy_szamu_routert_fertozott_meg_vilagszerte
%https://www.exploit-db.com/exploits/19185
%Auth bypass

\end{document}
