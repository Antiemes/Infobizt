\documentclass[12 pt]{beamer}
\usetheme[
  bullet=circle,		% Other option: square
  bigpagenumber,		% circled page number on lower right
  topline=true,			% colored bar at the top of the frame 
  shadow=false,			% Shading for beamer blocks
  watermark=BG_lower,	% png file for the watermark
]{Flip}


\newcommand{\titleimage}{title}			% Custom title 
\newcommand{\tanedo}{tanedolight}		% Custom author name
\newcommand{\CMSSMDM}{CMSSMDMlight.png}	% light background plot


%%%%%%%%%%
% FONTS %
%%%%%%%%%%

\usepackage[T1]{fontenc}
%\usepackage{lmodern}		
%\usepackage{sfmath}		% Sans Serif Math, off by default

%% Protects fonts from Beamer screwing with them
%% http://tex.stackexchange.com/questions/10488/force-computer-modern-in-math-mode
\usefonttheme{professionalfonts}


\usepackage[no-math]{fontspec}		

%\defaultfontfeatures{Mapping=tex-text}	% This seems to be important for mapping glyphs properly

\usepackage{amsmath}
%\usepackage{amsfonts}
%\usepackage{amssymb}
\usepackage{mathspec}
\usepackage{graphicx}
%\usepackage{mathrsfs} 			% For Weinberg-esque letters
\usepackage{cancel}				% For "SUSY-breaking" symbol
\usepackage{slashed}            % for slashed characters in math mode
%\usepackage{bbm}                % for \mathbbm{1} (unit matrix)
\usepackage{amsthm}				% For theorem environment
\usepackage{multirow}			% For multi row cells in table
\usepackage{arydshln} 			% For dashed lines in arrays and tables
\usepackage{tikzfeynman}		% For Feynman diagrams
% \usepackage{subfig}           % for sub figures
% \usepackage{young}			% For Young Tableaux
% \usepackage{xspace}			% For spacing after commands
% \usepackage{wrapfig}			% for Text wrap around figures
% \usepackage{framed}

\usepackage{setspace}

\setsansfont{calibri}[ 
Extension = .ttf,
UprightFont = *,
BoldFont = *b,
ItalicFont = *i,
Scale = 1
]

\setmathfont(Digits,Latin,Greek){SitkaI.ttc}

\graphicspath{{images/}}	% Put all images in this directory. Avoids clutter.


\usetikzlibrary{backgrounds}
\usetikzlibrary{mindmap,trees}	% For mind map
\usetikzlibrary{arrows,positioning,calc}
\usetikzlibrary{shapes}
% http://www.texample.net/tikz/examples/computer-science-mindmap/


% SOME COMMANDS THAT I FIND HANDY
% \renewcommand{\tilde}{\widetilde} % dinky tildes look silly, dosn't work with fontspec
\newcommand{\comment}[1]{\textcolor{comment}{\footnotesize{#1}\normalsize}} % comment mild
\newcommand{\Comment}[1]{\textcolor{Comment}{\footnotesize{#1}\normalsize}} % comment bold
\newcommand{\COMMENT}[1]{\textcolor{COMMENT}{\footnotesize{#1}\normalsize}} % comment crazy bold
\newcommand{\Alert}[1]{\textcolor{Alert}{#1}} % louder alert
\newcommand{\ALERT}[1]{\textcolor{ALERT}{#1}} % loudest alert
%% "\alert" is already a beamer pre-defined



\author{}
\title{Informatikai Rendszerek Biztonságtechnikája}
\institute{}
\date{}



\begin{document}

%% To use external nodes; http://www.texample.net/tikz/examples/beamer-arrows/
\tikzstyle{every picture}+=[remember picture]

{
  \setbeamertemplate{sidebar right}{\llap{\includegraphics[width=\paperwidth,height=\paperheight]{backgnd}}}

  \begin{frame}[c]
    \begin{center}
      % \includegraphics[width=7cm]{WarpedPenguinsReturn}

      \Large
      \textbf{Információs rendszerek}

      \textbf{biztonságtechnikája}

      \qquad

      Sebezhetőségek

      \qquad

      \textit{Vakulya Gergely}

    \end{center}
  \end{frame}
}

  %------------------------------------------------

\begin{frame}{A sebezhetőségek}
  \begin{itemize}
    \item{A sebezhetőség bizonyos rendszerek, szoftverek gyengesége, amit egy támadó a saját céljaira, illetve károkozásra tud kihasználni.}
    \item{A sebezhetőséget kihasználó szoftver az exploit.}
    \item{A következők időbeli alakulása lényegében tetszőlegesen alakulhat:}
      \begin{itemize}
        \item{A sebezhetőségek fennállásának időintervalluma}
        \item{Azok felfedezése}
        \item{A javítások elkészítése (a sebezhetőség befoltozása)}
        \item{A sebezhető rendszerek tényleges javítása}
        \item{A sebezhetőség nyilvánosságra hozása}
        \item{A sebezhetőség ismert, vagy feltételezett kihasználása}
      \end{itemize}
  \end{itemize}
\end{frame}

\begin{frame}{A CVE adatbázis}
  \begin{itemize}
    \item{CVE: Common Vulnerabilities and Exposures}
    \item{Nyilvános sérülékenység-adatbázis.}
    \item{A publikált sérülékenységek egy CVE számot kapnak, amiről egyértelműen be lehet őket azonosítani (emellett lehet fantázianevük is).}
    \item{Legfontosabb információk:}
      \begin{itemize}
        \item{Milyen szoftvert (vagy egyéb rendszert érint)}
        \item{Milyen verziókat érint}
        \item{Rögzítés dátuma}
        \item{Leírás (működési mechanizmus)}
        \item{Referenciák (például az egyes szállítók információihoz)}
      \end{itemize}
        \item{Több gyűjtőoldal:}
          \begin{itemize}
            \item{\href{https://cve.mitre.org}{https://cve.mitre.org}}
            \item{\href{https://www.cvedetails.com}{https://www.cvedetails.com}}
            \item{\href{https://www.cve.org/}{https://www.cve.org/}}
            \item{\href{https://nvd.nist.gov}{https://nvd.nist.gov}}
          \end{itemize}
  \end{itemize}
\end{frame}

\begin{frame}{A sebezhetőségek fő típusai}
  \begin{itemize}
    \item{Kódfuttatás (\textbf{code execution}): A támadó (esetenként tetszőleges) kódot képes futtatni a sebezhető rendszeren.}
    \item{Védelem, vagy ellenőrzés megkerülése (\textbf{bypass}): A támadó valamilyen ellenőrzés kihagyására képes a sebezhető rendszert késztetni.}
    \item{Privilégiumszint emelés (\textbf{privilege escalation}): A támadó egy számára nem biztosított privilégiumhoz (például root joghoz) jut.}
    \item{Szolgáltatásmegtagadás (\textbf{denial of service, DoS}): A támadó a rendszer szolgáltatásait leállítja, lelassítja, vagy azokat valamilyen módon akadályozza.}
    \item{Adatszivárgás (\textbf{information leak}): A támadó számára nem nyilvános adatokhoz jut.}
  \end{itemize}
\end{frame}

\begin{frame}{Támadási mechanizmusok}
  \begin{itemize}
    \item{Puffertúlcsordulásos támadás (buffer overflow, stack buffer overflow)}
    \item{SQL-injektálás (SQL injection, SQLi)}
    \item{Cross-side scripting, XSS}
    \item{Kéréshamisítás (Cross-site request forgery, CSRF vagy XSRF)}
    \item{Versenyhelyzet (Race condition)}
    \item{Sidechannel attack}
    \item{Memory corruption}
    \item{Bemenet ellenőrzés (Input validation)}
    \item{Kriptográfiai támadások}
  \end{itemize}

  \qquad

  Ezek kombinációi is gyakoriak.
\end{frame}

\begin{frame}{CVE-2019-18634, sudo pwfeedback exploit (Linux és egyéb rendszerek)}
  \begin{itemize}
    \item{A sudo 1.7.1 (2009-04-11) és 1.8.26 (2018-11-13) verziói között.}
    \item{A \texttt{sudo} parancs a jelszó bekérésekor nem ad kimenetet.}
    \item{Azonban beállíthatjuk, hogy a jelszó begépelése közben csillagok jelenjenek meg.}
    \item{Ha ez a beállítás fennáll, megfelelően összeállított bemenet beírásával a támadó root jogosultsághoz juthat.}
    \item{Típus: Helyi privilégiumszint növelés (local root).}
    \item{Mechanizmus: Stack buffer overflow.}
    \item{Megjegyzés: A `sudo` root joggal fut.}
  \end{itemize}
\end{frame}

%\begin{frame}{CVE-2021-3156, sudo}
%\end{frame}

\begin{frame}{CVE-2016-5195, Dirty COW (Linux)}
  \begin{itemize}
    \item{Linux kernel, 2.x-4.x, a 4.8.3 előtt.}
    \item{COW jelentése: Copy On Write}
    \item{Versenyhelyzet (race condition) alapú sebezhetőség.}
    \item{A kihasználásával a támadó egy olyan, a root tulajdonában levő fájlt tud írni, amire csak olvasási joga van.}
      \begin{itemize}
        \item{Átírhat konfigurációs fájlokat, például jogosultságokat adva magának.}
        \item{Backdoort helyezhet el valamelyik setuid root-os programban.}
      \end{itemize}
  \end{itemize}
\end{frame}

\begin{frame}{A Dirty COW működése}
  \begin{itemize}
    \item{Unix alatt ,,minden file''. A \texttt{/proc} alatt például virtuális fájlok vannak. A \texttt{/proc/self/mem} az aktuális folyamat memóriájának leképezése.}
    \item{\texttt{mmap()} függvény: Egy fájlt a memóriára képez le.}
    \item{Saját COW másolatot is készíthetünk egy fájlról.}
    \item{Copy on write: Csak akkor másolja le ténylegesen, ha módosítjuk.}
    \item{Race condition: Két eseménynak egymáshoz képest speciálisan időzítve kell lefutnia a hibához.}
    \item{Két thread-et futtatunk:}
      \begin{itemize}
        \item{Az egyikben az \texttt{madvice()} függvénnyel azt mondjuk a kernelnek, hogy az adott területet nem fogjuk a közeljövőben használni.}
        \item{A másikban a saját memóriát leképező \texttt{/proc/self/mem} azon részére írunk, ahova a támadott fájlt map-eltük.}
      \end{itemize}
    \item{Nagyon sok próbálkozás után egy bizonyos ponton a kernel ,,elrontja'' az írás és a másolás sorrendjét és felülíródik a megnyitt fájl.}
  \end{itemize}
\end{frame}

\begin{frame}{CVE-2017-5754, Meltdown (operációs rendszertől független)}
\end{frame}

\begin{frame}{CVE-2017-0199, Office arbitrary code execution}
\end{frame}

\begin{frame}{CVE-2017-0144, Eternal Blue}
\end{frame}

\begin{frame}{Slow loris (CVE-2007-6750 és sok egyéb)}
\end{frame}

\begin{frame}{CVE-2020-0796 aka CoronaBlue aka SMBGhost}
\end{frame}

\end{document}
