\documentclass[12 pt]{beamer}
\usetheme[
  bullet=circle,		% Other option: square
  bigpagenumber,		% circled page number on lower right
  topline=true,			% colored bar at the top of the frame 
  shadow=false,			% Shading for beamer blocks
  watermark=BG_lower,	% png file for the watermark
]{Flip}


\newcommand{\titleimage}{title}			% Custom title 
\newcommand{\tanedo}{tanedolight}		% Custom author name
\newcommand{\CMSSMDM}{CMSSMDMlight.png}	% light background plot


%%%%%%%%%%
% FONTS %
%%%%%%%%%%

\usepackage[T1]{fontenc}
%\usepackage{lmodern}		
%\usepackage{sfmath}		% Sans Serif Math, off by default

%% Protects fonts from Beamer screwing with them
%% http://tex.stackexchange.com/questions/10488/force-computer-modern-in-math-mode
\usefonttheme{professionalfonts}


\usepackage[no-math]{fontspec}		

%\defaultfontfeatures{Mapping=tex-text}	% This seems to be important for mapping glyphs properly

\usepackage{amsmath}
%\usepackage{amsfonts}
%\usepackage{amssymb}
\usepackage{mathspec}
\usepackage{graphicx}
%\usepackage{mathrsfs} 			% For Weinberg-esque letters
\usepackage{cancel}				% For "SUSY-breaking" symbol
\usepackage{slashed}            % for slashed characters in math mode
%\usepackage{bbm}                % for \mathbbm{1} (unit matrix)
\usepackage{amsthm}				% For theorem environment
\usepackage{multirow}			% For multi row cells in table
\usepackage{arydshln} 			% For dashed lines in arrays and tables
\usepackage{tikzfeynman}		% For Feynman diagrams
% \usepackage{subfig}           % for sub figures
% \usepackage{young}			% For Young Tableaux
% \usepackage{xspace}			% For spacing after commands
% \usepackage{wrapfig}			% for Text wrap around figures
% \usepackage{framed}

\usepackage{setspace}

\setsansfont{calibri}[ 
Extension = .ttf,
UprightFont = *,
BoldFont = *b,
ItalicFont = *i,
Scale = 1
]

\setmathfont(Digits,Latin,Greek){SitkaI.ttc}

\graphicspath{{images/}}	% Put all images in this directory. Avoids clutter.


\usetikzlibrary{backgrounds}
\usetikzlibrary{mindmap,trees}	% For mind map
\usetikzlibrary{arrows,positioning,calc}
\usetikzlibrary{shapes}
% http://www.texample.net/tikz/examples/computer-science-mindmap/


% SOME COMMANDS THAT I FIND HANDY
% \renewcommand{\tilde}{\widetilde} % dinky tildes look silly, dosn't work with fontspec
\newcommand{\comment}[1]{\textcolor{comment}{\footnotesize{#1}\normalsize}} % comment mild
\newcommand{\Comment}[1]{\textcolor{Comment}{\footnotesize{#1}\normalsize}} % comment bold
\newcommand{\COMMENT}[1]{\textcolor{COMMENT}{\footnotesize{#1}\normalsize}} % comment crazy bold
\newcommand{\Alert}[1]{\textcolor{Alert}{#1}} % louder alert
\newcommand{\ALERT}[1]{\textcolor{ALERT}{#1}} % loudest alert
%% "\alert" is already a beamer pre-defined



\author{}
\title{Informatikai Rendszerek Biztonságtechnikája}
\institute{}
\date{}



\begin{document}

%% To use external nodes; http://www.texample.net/tikz/examples/beamer-arrows/
\tikzstyle{every picture}+=[remember picture]

{
  \setbeamertemplate{sidebar right}{\llap{\includegraphics[width=\paperwidth,height=\paperheight]{backgnd}}}

  \begin{frame}[c]
    \begin{center}
      % \includegraphics[width=7cm]{WarpedPenguinsReturn}

      \Large
      \textbf{Információs rendszerek}

      \textbf{biztonságtechnikája}

      \qquad

      Social engineering

      \qquad

      \textit{Vakulya Gergely}

    \end{center}
  \end{frame}
}

  %------------------------------------------------

\begin{frame}{A social engineering}
  \begin{itemize}
    \item{Az emberi tényező kihasználható tulajdonságaira építő támadási forma}
    \item{Az emberek befolyásolására, manipulálására alapoz}
    \item{Lehetséges célok:}
      \begin{itemize}
        \item{Bizalmas információk megszerzése}
        \item{Szándékos károkozás}
      \end{itemize}
    \item{\textit{A social engineering a befolyásolás és a rábeszélés eszközével megtéveszti az embereket, manipulálja, vagy meggyőzi őket, hogy a social engineer tényleg az, akinek mondja magát. Ennek eredményeként a social engineer – technológia használatával vagy anélkül – képes az embereket információszerzés érdekében kihasználni.} -- Kevin Mitnick}
  \end{itemize}
\end{frame}

%------------------------------------------------

\begin{frame}{Miért hatékony a social engineering?}
  \begin{itemize}
    \item{A munkavállaló a legtöbb védendő értékhez közvetlenül hozzáférheti}
      \begin{itemize}
        \item{Hardver}
        \item{Szoftver}
        \item{Adatok (például fejlesztési, üzleti, kontakt stb.)}
      \end{itemize}
    \item{Ez a fajta támadás nem a (fizikai, vagy logikai) biztonsági
      rendszert, hanem a humán erőforrást támadja.}
    \item{A munkavállalók nem tulajdonítanak kellő jelentőséget a rájuk
      bizott adatoknak. Önmagukban, kontextus nélkül nem értelmezhetők.}
      \begin{itemize}
        \item{Felhasználónév jelszó nélkül}
        \item{Jelszó felhasználónév nélkül}
        \item{Akár felhasználónév és jelszó együtt, amivel \textit{úgysem tud mit kezdeni}}
        \item{Születési idők, becenevek, hobbik, gyerekek, házikedvencek nevei.}
        \item{Infrastruktúrára vonatkozó adatok (gépek nevei, számozási sémák stb.)}
      \end{itemize}
  \end{itemize}
\end{frame}

%------------------------------------------------

\begin{frame}{Social engineering módszerek}
  Az áldozat megtévesztéséhez, a ráhatáshoz nincsen szükség számítógép használatára.

  \begin{itemize}
    \item{Telefonhívás: A partner nehezen tudja a támadót azonosítani}
      \begin{itemize}
        \item{Segítség kérése, főleg ha sürgős}
        \item{Felettesnek adja ki magát}
        \item{Látszólag lényegtelen dolgokat kérdez statisztikai céllal}
        \item{Adategyeztetés}
      \end{itemize}
    \item{Személyes behatolás: Bizonyos feltételek esetén (nem túl kicsi vállalat, de nincs belépőkártya, vagy kód)}
      \begin{itemize}
        \item{Új alkalmazott}
        \item{Látogató}
        \item{Ügyfél}
        \item{Eltévedt}
        \item{Karbantartó}
        \item{Valakihez jött}
      \end{itemize}
  \end{itemize}

\end{frame}

%------------------------------------------------

\begin{frame}{Két klasszikus példa}
Piggybacking

  \begin{itemize}
    \item{Valakihez csatlakozva hatol be a támadó.}
    \item{Pl. otthon hagyott belépőkártya, vagy épp nem becsukódó ajtó}
  \end{itemize}

Tailgating

  \begin{itemize}
    \item{A támadó egy csoporthoz csatlakozik, majd leválik}
  \end{itemize}

\end{frame}

%------------------------------------------------

\begin{frame}{Esettanulmány: CVE-2024-3094 (XZ-Utils SSH backdoor}
\end{frame}

%berner, würth

\end{document}
